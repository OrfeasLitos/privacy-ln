\section{Payment Network Functionality}
  \label{sec:payfunc}
  \begin{figure}[H]
    \begin{systembox}{\fpaynet -- interface}
      \begin{itemize}
        \item from $\environment$:
        \begin{itemize}
          \item (\textsc{register}, delay, relayDelay)
          \item (\textsc{toppedUp})
          \item (\textsc{openChannel}, \alice, \bob, $x$, \textit{tid})
          \item (\textsc{checkForNew}, \alice, \bob, \textit{tid})
          \item (\textsc{pay}, \bob, $x, \overrightarrow{\mathtt{path}}$,
          \texttt{receipt})
          \item (\textsc{closeChannel}, \texttt{receipt}, \textit{pchid})
          \item (\textsc{forceCloseChannel}, \texttt{receipt}, \textit{pchid})
          \item (\textsc{poll}) {\color{blue} - obsolete}
          \item (\textsc{pushFulfill}, \textit{pchid}) {\color{blue} - obsolete}
          \item (\textsc{pushAdd}, \textit{pchid}) {\color{blue} - obsolete}
          \item (\textsc{commit}, \textit{pchid}) {\color{blue} - obsolete}
          \item (\textsc{fulfillOnChain}) {\color{blue} - obsolete}
          \item (\textsc{getNews})
        \end{itemize}
        \item to $\environment$:
        \begin{itemize}
          \item (\textsc{register}, \alice, \texttt{delay}(\alice),
          \texttt{relayDelay}(\alice), pubKey)
          \item (\textsc{registered})
          \item (\textsc{news}, newChannels, closedChannels, updatesToReport)
        \end{itemize}
        \item from $\simulator$:
        \begin{itemize}
          \item (\textsc{registerDone}, \alice, pubKey)
          \item (\textsc{channelAnnounced}, \alice, $p_{\alice, F}, p_{\bob,
          F}$, \textit{fchid}, \textit{pchid}, \textit{tid})
          \item (\textsc{update}, \texttt{receipt}, \alice) {\color{blue} -
          obsolete}
          \item (\textsc{closedChannel}, \texttt{channel}, \alice)
          \item (\textsc{resolvePays}, \textit{payid}, \texttt{charged})
          {\color{blue} - obsolete}
        \end{itemize}
        \item to $\simulator$:
        \begin{itemize}
          \item (\textsc{register}, \alice, delay, relayDelay)
          \item (\textsc{openChannel}, \alice, \bob, $x$, \textit{fchid},
          \textit{tid})
          \item (\textsc{channelOpened}, \alice, \textit{fchid})
          \item (\textsc{pay}, \alice, \bob, $x,
          \overrightarrow{\mathtt{path}}$, \texttt{receipt}, \textit{payid})
          {\color{blue} - obsolete}
          \item (\textsc{continue}) {\color{blue} - obsolete}
          \item (\textsc{closeChannel}, \textit{fchid}, \alice)
          \item (\textsc{forceCloseChannel}, \textit{fchid}, \alice)
          \item (\textsc{poll}, $\Sigma_{\alice}$, \alice) {\color{blue} -
          obsolete}
          \item (\textsc{pushFulfill}, \textit{pchid}, \alice) {\color{blue} -
          obsolete}
          \item (\textsc{pushAdd}, \textit{pchid}, \alice) {\color{blue} -
          obsolete}
          \item (\textsc{commit}, \textit{pchid}, \alice) {\color{blue} -
          obsolete}
          \item (\textsc{fulfillOnChain}, $t$, \alice) {\color{blue} - obsolete}
        \end{itemize}
      \end{itemize}
    \end{systembox}
    \caption{}
    \label{alg:fpaynet:interface}
  \end{figure}

  \maybeshow{
  All players need to register in order to use channels. The registration of
  \alice works as follows: \alice inputs her desired delay and relayDelay
  that will be used for all her future channels. The first denotes how often she
  has to check the blockchain for revoked commitments and the second defines the
  minimum time distance between incoming and outgoing CLTV expiries. \fpaynet
  then informs \simulator, who sends back a long-lived public key for \alice.
  This key represents \alice's account, from where \fpaynet can get coins to
  open new channels on her behalf and to place coins of closed channels. The key
  is sent to \alice who moves some initial funds to it and notifies
  \fpaynet. She is now registered. The exact logic is found in
  Fig.~\ref{alg:fpaynet:support}, which also contains the actions of \fpaynet
  related to corruptions.

  Additionally, the procedure \texttt{checkClosed}() is called after
  \textsc{read}ing from \ledger, with the received state $\Sigma$ as input. This
  call happens every time \fpaynet \textsc{read}s from \ledger. The formal
  definition of \texttt{checkClosed}() can be found in
  Fig.~\ref{alg:fpaynet:close:func}, along with a discussion of its purpose.
  }

  \begin{figure}[H]
    \begin{systembox}{\fpaynet -- registration and corruption}
      \begin{algorithmic}[1]
        \State Initialisation:
        \Indent
          \State $\mathtt{channels}, \mathtt{pendingPay}, \mathtt{pendingOpen},
          \mathtt{corrupted}, \Sigma \gets \emptyset$
        \EndIndent
        \Statex

        \State Upon receiving $\left(\textsc{register}, \mathrm{delay},
        \mathrm{relayDelay}\right)$ from \alice:
        \Indent
          \State $\mathtt{delay}\left(\alice\right) \gets \mathrm{delay}$
          \Comment{Must check chain at least once every \texttt{delay}(\alice)
          blocks}
          \State $\mathtt{relayDelay}\left(\alice\right) \gets
          \mathrm{relayDelay}$
          \State $\mathtt{updatesToReport}\left(\alice\right),
          \mathtt{newChannels}\left(\alice\right) \gets \emptyset$
          \State $\mathtt{polls}\left(\alice\right) \gets \emptyset$
          \State $\mathtt{focs}\left(\alice\right) \gets \emptyset$
          \State send (\textsc{read}) to \ledger as \alice, store reply to
          $\Sigma_{\alice}$, add $\Sigma_{\alice}$ to $\Sigma$ and add largest
          block number to \texttt{polls}(\alice)
          \label{alg:fpaynet:support:read}
          \State \texttt{checkClosed}($\Sigma_{\alice}$)
          \State send $\left(\textsc{register}, \alice, \mathrm{delay},
          \mathrm{relayDelay}\right)$ to \simulator
        \EndIndent
        \Statex

        \State Upon receiving $\left(\textsc{registerDone}, \alice,
        \mathrm{pubKey}\right)$ from \simulator:
        \Indent
          \State $\mathtt{pubKey}\left(\alice\right) \gets \mathrm{pubKey}$
          \State send (\textsc{register}, \alice, \texttt{delay}(\alice),
          \texttt{relayDelay}(\alice), pubKey) to \alice
        \EndIndent
        \Statex

        \State Upon receiving (\textsc{toppedUp}) from \alice:
        \Indent
          \State send (\textsc{read}) to \ledger as \alice and store reply
          to $\Sigma_{\alice}$
          \State \texttt{checkClosed}($\Sigma_{\alice}$)
          \State assign the sum of all output values that are exclusively
          spendable by \alice to \texttt{onChainBalance}
          \State send (\textsc{registered}) to \alice
        \EndIndent
        \Statex

        \State Upon receiving any message ($M$) except for (\textsc{register})
        or (\textsc{toppedUp}) from \alice:
        \Indent
          \If{if haven't received (\textsc{register}) and (\textsc{toppedUp})
          from \alice (in this order)}
            \State send (\textsc{invalid}, $M$) to \alice and ignore message
          \EndIf
          \label{alg:fpaynet:support:unreg}
        \EndIndent
      \end{algorithmic}
    \end{systembox}
    \caption{}
    \label{alg:fpaynet:support}
  \end{figure}

  \maybeshow{
  The process of \alice opening a channel with \bob is as follows: First
  \alice asks \fpaynet to open and \fpaynet informs \simulator.
  \simulator provides the necessary keys and IDs for the new channel to
  \fpaynet. \alice asks \fpaynet to check if \ledger contains the funding
  transaction from \alice's point of view. If it does, \fpaynet activates
  \simulator, who in turn returns control to \fpaynet. Now \fpaynet checks
  that the funding transaction is in the \ledger also from \bob's point of
  view and in case it does, it notifies \simulator. \simulator then confirms
  that to \fpaynet that the channel is open and \fpaynet finally stores the
  channel as open. This last exchange is needed to match the real-world
  interaction.
  }

  \begin{figure}[H]
    \begin{systembox}{\fpaynet -- open}
      \begin{algorithmic}[1]
        \State Upon receiving $\left(\textsc{openChannel}, \alice, \bob, x,
        \mathit{tid}\right)$ from \alice:
        \Indent
          \State ensure \textit{tid} hasn't been used by \alice for opening
          another channel before
          \label{alg:fpaynet:open:valid}
          \State choose unique channel ID \textit{fchid}
          \State $\mathtt{pendingOpen}\left(\mathit{fchid}\right) \gets
          \left(\alice, \bob, x, \mathit{tid}\right)$
          \State send $\left(\textsc{openChannel}, \alice, \bob, x,
          \mathit{fchid}, \mathit{tid}\right)$ to \simulator
        \EndIndent
        \Statex

        \State Upon receiving (\textsc{channelAnnounced}, \alice, $p_{\alice,
        F}$, $p_{\bob, F}$, \textit{fchid}, \textit{pchid}, \textit{tid}) from
        \simulator:
        \Indent
          \State ensure that there is a \texttt{pendingOpen}(\textit{fchid})
          entry with temporary id \textit{tid}
          \label{alg:fpaynet:announced:valid}
          \State add $p_{\alice, F}, p_{\bob, F}$, \textit{pchid} and  mark
          ``\alice announced'' to \texttt{pendingOpen}(\textit{fchid})
          \label{alg:fpaynet:announced:add}
        \EndIndent
        \Statex

        \State Upon receiving (\textsc{checkForNew}, \alice, \bob, \textit{tid})
        from \alice:
        \Indent
          \State ensure there is a matching \texttt{channel} in
          \texttt{pendingOpen}(\textit{fchid}), marked with ``\alice
          announced''
          \label{alg:fpaynet:checkForNew:valid}
          \State $\left(\mathrm{funder}, \mathrm{fundee}, x, p_{\alice, F},
          p_{\bob, F}\right) \gets
          \mathtt{pendingOpen}\left(\mathit{fchid}\right)$
          \State send (\textsc{read}) to \ledger as \alice and store reply
          to $\Sigma_{\alice}$
          \State \texttt{checkClosed}($\Sigma_{\alice}$)
          \label{alg:fpaynet:checkForNew:read:alice}
          \State ensure that there is a TX $F \in \Sigma_{\alice}$ with a
          $\left(x, \left(p_{\mathrm{funder}, F} \wedge p_{\mathrm{fundee},
          F}\right)\right)$ output
          \label{alg:fpaynet:checkForNew:included}
          \State mark \texttt{channel} with ``waiting for
          \textsc{fundingLocked}''
          \label{alg:fpaynet:checkForNew:mark}
          \State send (\textsc{fundingLocked}, \alice, $\Sigma_{\alice}$,
          \textit{fchid}) to \simulator
          \label{alg:fpaynet:checkForNew:sim}
        \EndIndent
        \Statex

        \State Upon receiving (\textsc{fundingLocked}, \textit{fchid}) from
        \simulator:
        \Indent
          \State ensure a \texttt{channel} is in
          \texttt{pendingOpen}(\textit{fchid}), marked with ``waiting for
          \textsc{fundingLocked}'' and replace mark with ``waiting for
          \textsc{channelOpened}''
          \State send (\textsc{read}) to \ledger as \bob and store reply
          to $\Sigma_{\bob}$
          \State \texttt{checkClosed}($\Sigma_{\bob}$)
          \label{alg:fpaynet:checkForNew:read:bob}
          \State ensure that there is a TX $F \in \Sigma_{\bob}$ with a
          $\left(x, \left(p_{\mathrm{funder}, F} \wedge p_{\mathrm{fundee},
          F}\right)\right)$ output
          \State add \texttt{receipt}(\texttt{channel}) to
          \texttt{newChannels}(\bob)
          \label{alg:fpaynet:fundingLocked:report}
          \State send (\textsc{fundingLocked}, \bob, $\Sigma_{\bob}$,
          \textit{fchid}) to \simulator
          \label{alg:fpaynet:fundingLocked:sim}
        \EndIndent
        \Statex

        \State Upon receiving (\textsc{channelOpened}, \textit{fchid}) from
        \simulator:
        \Indent
          \State ensure a \texttt{channel} is in
          \texttt{pendingOpen}(\textit{fchid}), marked with ``waiting for
          \textsc{channelOpened}'' and remove mark
          \State $\mathrm{offChainBalance}\left(\mathrm{funder}\right) \gets
          \mathrm{offChainBalance}\left(\mathrm{funder}\right) + x$
          \label{alg:fpaynet:channelOpened:offchain}
          \State $\mathrm{onChainBalance}\left(\mathrm{funder}\right) \gets
          \mathrm{onChainBalance}\left(\mathrm{funder}\right) - x$
          \label{alg:fpaynet:channelOpened:onchain}
          \State $\mathtt{channel} \gets \left(\mathrm{funder}, \mathrm{fundee},
          x, 0, 0, \mathit{fchid}, \mathit{pchid}\right)$
          \State add \texttt{channel} to \texttt{channels}
          \State add \texttt{receipt}(\texttt{channel}) to
          \texttt{newChannels}(\alice)
          \label{alg:fpaynet:channelOpened:report}
          \State clear \texttt{pendingOpen}(\textit{fchid}) entry
        \EndIndent
      \end{algorithmic}
    \end{systembox}
    \caption{}
    \label{alg:fpaynet:open}
  \end{figure}

  \maybeshow{
  When instructed to perform a payment, \fpaynet simply takes note of the
  message and forwards it to \simulator. It also remembers to inform the payer
  that the payment has been completed when \simulator says so. Observe here
  that \fpaynet trusts \simulator to correctly carry out channel updates.
  While counterintuitive, it allows \fpaynet to ignore the details of channel
  updates, signatures, key and transaction handling. Nevertheless, as we will
  see \fpaynet keeps track of requested and ostensibly carried out updates and
  ensures that upon channel closure the balances are as expected, therefore
  ensuring funds security.
  }

  \begin{figure}[H]
    \begin{systembox}{\fpaynet -- pay {\color{blue} (updated)}}
      \begin{algorithmic}[1]
        \State Upon receiving $\left(\textsc{pay}, \bob, x,
        \overrightarrow{\mathtt{path}}\right)$ from \alice:
        \Indent
          \State ensure that $\overrightarrow{\mathtt{path}}$ consists of open
          channels that form a path of capacity at least $x$ (in the right
          direction) from \alice to \bob
          \State starting on $|\overrightarrow{\mathtt{path}}|$ clock ticks after
          receiving this message, on every clock tick, channel $\in
          \overrightarrow{\mathtt{path}}$, reduce balance of party closer to payer
          by $x$ and increase balance of party closer to payee by $x$ in the next
          channel on the $\overrightarrow{\mathtt{path}}$ and add \texttt{receipt}
          of new balance to both parties' \texttt{updatesToReport}, starting from
          the unique channel in which the payee is participating
        \EndIndent
      \end{algorithmic}
    \end{systembox}
    \caption{}
    \label{alg:fpaynet:pay}
  \end{figure}

  \maybeshow{
  Similarly to payment instructions, when \fpaynet receives a message
  instructing it to close a channel (Fig.~\ref{alg:fpaynet:close}), it takes a
  note of the pending closure, it stops serving any more requests for this
  channel and it forwards the request to \simulator. In turn \simulator
  notifies \fpaynet of a closed channel with the corresponding message, upon
  which \fpaynet takes a note to inform the corresponding player. Depending on
  whether the message instructed for a unilateral or a cooperative close,
  \fpaynet will either put or not a time limit respectively to the service of
  the request. In particular, in case of cooperative close, the time limit is
  infinity (l.~\ref{alg:fpaynet:close:coop:mark}). As we will see, in case a
  unilateral close request was made and the time limit for servicing it is
  reached, \fpaynet halts (Fig.~\ref{alg:fpaynet:close:func},
  l.~\ref{alg:fpaynet:close:func:idle}). Once more \fpaynet trusts \simulator,
  but later checks that the chain contains the correct transactions with
  \texttt{checkClosed}() (Fig.~\ref{alg:fpaynet:close:func}).
  }

  \begin{figure}[H]
    \begin{systembox}{\fpaynet -- close}
      \begin{algorithmic}[1]
        \State Upon receiving (\textsc{closeChannel}, \texttt{receipt},
        \textit{pchid}) from \alice
        \Indent
          \State ensure that there is a $\mathtt{channel} \in \mathtt{channels}
          : \mathtt{receipt}\left(\mathtt{channel}\right) = \mathtt{receipt}$
          with ID \textit{pchid}
          \label{alg:fpaynet:close:coop:ensure}
          \State retrieve \textit{fchid} from \texttt{channel}
          \label{alg:fpaynet:close:coop:retrieve}
          \State add (\textit{fchid}, \texttt{receipt}(\texttt{channel}),
          $\infty$) to \texttt{pendingClose}(\alice)
          \label{alg:fpaynet:close:coop:mark}
          \State do not serve any other (\textsc{pay}, \textsc{closeChannel})
          message from \alice for this channel
          \label{alg:fpaynet:close:coop:noserve}
          \State send (\textsc{closeChannel}, \texttt{receipt}, \textit{pchid},
          \alice) to \simulator
          \label{alg:fpaynet:close:coop:send}
        \EndIndent
        \Statex

        \State Upon receiving (\textsc{forceCloseChannel}, \texttt{receipt},
        \textit{pchid}) from \alice
        \Indent
          \State retrieve \textit{fchid} from \texttt{channel}
          \label{alg:fpaynet:close:unilateral:retrieve}
          \State add (\textit{fchid}, \texttt{receipt}(\texttt{channel}),
          $\bot$) to \texttt{pendingClose}(\alice)
          \label{alg:fpaynet:close:unilateral:mark}
          \State do not serve any other (\textsc{pay}, \textsc{closeChannel},
          \textsc{forceCloseChannel}) message from \alice for this channel
          \label{alg:fpaynet:close:unilateral:noserve}
          \State send (\textsc{forceCloseChannel}, \texttt{receipt},
          \textit{pchid}, \alice) to \simulator
          \label{alg:fpaynet:close:unilateral:send}
        \EndIndent
        \Statex

        \State Upon receiving (\textsc{closedChannel}, \texttt{channel}, \alice)
        from \simulator:
        \Indent
          \State remove any (\textit{fchid} of channel,
          \texttt{receipt}(\texttt{channel}), $\infty$) from
          \texttt{pendingClose}(\alice)
          \State add (\textit{fchid} of channel,
          \texttt{receipt}(\texttt{channel}), $\bot$) to
          \texttt{closedChannels}(\alice)
          \Comment{trust \simulator here, check on \texttt{checkClosed}()}
          \label{alg:fpaynet:closedChannel:report}
          \State send (\textsc{continue}) to \simulator
        \EndIndent
        \Statex
      \end{algorithmic}
    \end{systembox}
    \caption{}
    \label{alg:fpaynet:close}
  \end{figure}

  \maybeshow{
  After every \textsc{read} \fpaynet sends to \ledger and its response is
  received, \texttt{checkClosed}() (Fig.~\ref{alg:fpaynet:close:func}) is
  called. \fpaynet checks the input state $\Sigma$ for transactions that close
  channels and, in case no security violation has taken place, it updates the
  on- and off-chain balances of the player accordingly
  (ll.~\ref{alg:fpaynet:close:func:happy:start}-\ref{alg:fpaynet:close:func:happy:end}).
  The possible security violations are: signature forgery
  (l.~\ref{alg:fpaynet:close:func:dsforgery}), malicious closure even though the
  player was not negligent (l.~\ref{alg:fpaynet:close:func:malicious}), no
  closing transaction in $\Sigma$ even though the player asked for channel
  closure a substantial amount of time before
  (l.~\ref{alg:fpaynet:close:func:idle}) and incorrect on- or off-chain balance
  after the closing of all of the player's channels
  (l.~\ref{alg:fpaynet:close:func:balance}).
  }

  \begin{figure}[H]
    \begin{systembox}{\fpaynet -- \texttt{checkClosed}()}
      \begin{algorithmic}[1]
        \Function{checkClosed}{$\Sigma_{\alice}$} \Comment{Called after every
        (\textsc{read}), ensures requested closes eventually happen}
        \label{alg:fpaynet:close:func:start}
          \If{there is any closing/commitment transaction in $\Sigma_{\alice}$
          with no corresponding entry in $\mathtt{pendingClose}(\alice) \cup
          \mathtt{closedChannels}(\alice)$}
          \label{alg:fpaynet:close:func:ifunnotified}
            \State add $(\mathit{fchid}, \mathtt{receipt}, \bot)$ to
            \texttt{closedChannels}(\alice), where \textit{fchid} is the ID of
            the corresponding \texttt{channel}, \texttt{receipt} comes from the
            latest \texttt{channel} state
            \label{alg:fpaynet:close:func:unnotified}
          \EndIf
          \ForAll{entries $(\mathit{fchid}, \mathtt{receipt}, h) \in
          \mathtt{pendingClose}(\alice) \cup \mathtt{closedChannels}(\alice)$}
            \If{there is a closing/commitment transaction in $\Sigma_{\alice}$
            for open \texttt{channel} with ID \textit{fchid} with a balance that
            corresponds to \texttt{receipt}}
            \label{alg:fpaynet:close:func:happy:start}
              \State let $x, y$ \alice's and \texttt{channel} counterparty
              \bob's balances respectively
              \State $\mathrm{offChainBalance}\left(\alice\right) \gets
              \mathrm{offChainBalance}\left(\alice\right) - x$
              \State $\mathrm{onChainBalance}\left(\alice\right) \gets
              \mathrm{onChainBalance}\left(\alice\right) + x$
              \label{alg:fpaynet:close:func:alice:credit}
              \State $\mathrm{offChainBalance}\left(\bob\right) \gets
              \mathrm{offChainBalance}\left(\bob\right) - y$
              \State $\mathrm{onChainBalance}\left(\bob\right) \gets
              \mathrm{onChainBalance}\left(\bob\right) + y$
              \label{alg:fpaynet:close:func:bob:credit}
              \State remove \texttt{channel} from \texttt{channels} \& entry
              from \texttt{pendingClose}(\alice)
              \If{there is an (\textit{fchid}, \_, \_) entry in
              \texttt{pendingClose}(\bob)}
                \State remove it from \texttt{pendingClose}(\bob)
              \EndIf
              \label{alg:fpaynet:close:func:happy:end}
            \ElsIf{there is a tx in $\Sigma_{\alice}$ that is not a
            closing/commitment tx and spends the funding tx of the
            \texttt{channel} with ID \textit{fchid}}
            \label{alg:fpaynet:close:func:ifdsforgery}
              \State halt \Comment{DS forgery}
              \label{alg:fpaynet:close:func:dsforgery}
            \ElsIf{there is a commitment transaction in block of height $h$ in
            $\Sigma_{\alice}$ for open \texttt{channel} with ID \textit{fchid}
            with a balance that does not correspond to the \texttt{receipt} and
            the delayed output has been spent by the counterparty}
              \If{\texttt{polls}(\alice) contains an entry in $[h, h
              + \mathtt{delay}(\alice) - 1]$}
              \label{alg:fpaynet:close:func:ifmalicious}
                \State halt
                \label{alg:fpaynet:close:func:malicious}
              \Else
                \State $\mathtt{negligent}(\alice) \gets \true$
              \EndIf
            \ElsIf{there is no such closing/commitment transaction $\wedge \ h =
            \bot$}
              \State assign largest block number of $\Sigma_{\alice}$ to $h$ of
              entry
            \ElsIf{there is no such closing/commitment transaction $\wedge \ h
            \neq \bot \ \wedge$ (largest block number of $\Sigma_{\alice}) \geq
            h + \tochain$}
            \label{alg:fpaynet:close:func:ifidle}
              \State halt
              \label{alg:fpaynet:close:func:idle}
            \EndIf
          \EndFor
          \If{\alice has no open channels in $\Sigma_{\alice}$ AND
          $\mathtt{negligent}(\alice) = \false$}
            \If{$\mathrm{offChainBalance}(\alice) \neq 0$ OR
            onChainBalance(\alice) is not equal to the total funds exclusively
            spendable by \alice in $\Sigma_{\alice}$}
              \State halt
              \label{alg:fpaynet:close:func:balance}
            \EndIf
          \EndIf
        \EndFunction
        \label{alg:fpaynet:close:func:end}
      \end{algorithmic}
    \end{systembox}
    \caption{}
    \label{alg:fpaynet:close:func}
  \end{figure}

  \maybeshow{
  \textsc{getNews} requests from \fpaynet information on newly opened, closed
  and updated channels.
  }

  \begin{figure}[H]
    \begin{systembox}{\fpaynet -- get news {\color{blue} (updated)}}
      \begin{algorithmic}[1]
        \State Upon receiving (\textsc{getNews}) from \alice:
        \label{alg:fpaynet:getnews}
        \Indent
          \State clear \texttt{newChannels}(\alice),
          \texttt{closedChannels}(\alice), \texttt{updatesToReport}(\alice) and
          send them to \alice with message name \textsc{news}, stripping
          \textit{fchid} and h from \texttt{closedChannels}(\alice)
          \label{alg:fpaynet:getnews:send}
        \EndIndent
      \end{algorithmic}
    \end{systembox}
    \caption{}
    \label{alg:fpaynet:daemon}
  \end{figure}

  \begin{itemize}
    \item The functionality above provides unobservability of off-chain payments
    and can be realised by a protocol in which all players communicate with every
    other player on every round -- sending garbage to the ones with which they
    don't have to interact. Such a protocol has $n^2$ communication cost.
    Indistinguishability holds only in case of a global passive adversary (no
    corruptions).
    \item We can also assert unobservability for paths that consist of honest
    parties only in the case where there is a system-wide maximum path length $l$
    and corruptions activate only after $2l$ clock ticks. Both the functionality
    and the protocol would be the same.
    \item In case of a normal corruption model, the functionality has to leak the
    previous and next player on the path, along with the payment value, to a
    corrupted player that is to receive its message on this clock tick. Also, the
    functionality has to wait for confirmation from the corrupted player before
    sending the message to the next player (but this isn't strictly about
    privacy).
  \end{itemize}
